%        File: carpeta.tex
%     Created: lun may 01 03:00  2017 A
% Last Change: lun may 01 03:00  2017 A

\documentclass[a4paper]{article}
\oddsidemargin 0in
\textwidth 6.5in
\topmargin -1.5cm
\textheight 22cm
\usepackage[utf8]{inputenc}
\usepackage{amsmath}
\usepackage{amsfonts}
\usepackage{amssymb}
\usepackage{graphicx}
\usepackage{titlesec}
\usepackage{fancyhdr}
\pagestyle{fancy}
\headheight=30pt

\title{Carpeta de clases \\Software en Tiempo Real \\UTN frc}
\author{Martin Nievas}

\makeindex
\begin{document}
\maketitle
\pagebreak
\makeindex
\section{Introducción} % (fold)
\label{sec:Introduccion}
Este docuemento contiene las notas de clases de la materia dictada durante el 
año 2017 en la UTN - Facultad Regional Córdoba.\\No pretende ser un material 
de estudio solo es una forma de ordenar los temas que fuí tomando en la carpeta.

\subsection{Sección crítica} % (fold)
Es una región de código que no puede ser dividida, suelen desactivarse las 
interrupciones antes de entrar.
\subsection{tarea} % (fold)
\label{sub:Tarea}
Es una porción de código que pienza que toda la CPU está para ella sola.


\subsection{Cambio de contexto} % (fold)
\label{sub:Cambio de contexto}
Cuando el kernel decide correr una tarea diferente, almacena el contexto de la 
tarea en el área de almacenamiento de contexto.
% subsection Cambio de contexto (end)
\subsection{Kernel} % (fold)
\label{sub:Kernel}
Es una parte en un sistema multitarea encargado de la administración de las 
tareas y la comunicación entre estas. El servicio fundamental provisto por el 
kernel es el cambio de contexto
% subsection Kernel (end)
\subsection{Scheduler} % (fold)
\label{sub:Scheduler}
También conocido como el ``despachador" es la parte del kernel responsable de 
asignar que tarea sigue después de la actual. La mayoría de los kernel son
basados en prioridad.
% subsection Scheduler (end)
\end{document}

